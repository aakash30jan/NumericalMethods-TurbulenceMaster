\documentclass[10pt,a4paper]{article}
\usepackage[latin1]{inputenc}
\usepackage{amsmath}
\usepackage{amsfonts}
\usepackage{amssymb}
\usepackage{graphicx}
\author{Bipin Kumar Singh}
\title{NUMERICAL PRACTICE -1: NUMERICAL INTERPOLATION}

\usepackage[T1]{fontenc}
\usepackage{xcolor}
\usepackage{lmodern}
\usepackage{listings}
\lstset{language=[90]Fortran,
	basicstyle=\ttfamily,
	keywordstyle=\color{red},
	commentstyle=\color{green},
	morecomment=[l]{!\ }% Comment only with space after !
}


\begin{document}
	\maketitle

\section{THEORY:INTERPOLATION}
To determine intermediate value between precise data points.The most common method used for this purpose is polynomial interpolation.$$ f(x) = a0 + a1 *x+a2 *x ^2 +------ + an *x^ n$$

For $n + 1$ data points, there is one and only one polynomial of order n that passes through all the points. For example, there is only one straight line (that is, a first-order polynomial) that connects two points. Similarly, only one parabola connects a set of three points. Polynomial interpolation consists of determining the unique nth-order polynomial that fits $n + 1$ data points. This polynomial then provides a formula to compute
intermediate values.
\subsection{NEWTON'S DIVIDING DIFFERENCE INTERPOLATION METHOD}
There are a variety of alternative forms for expressing an interpolating
polynomial. Newton?s divided-difference interpolating polynomial is among the most popular and useful forms. 

The generalized equation to fit an nth-order polynomial to n + 1 data points.The nth-order polynomial as per Newton's method is:$$fn(x) = b0 + b1*(x-x 0 ) + -------- + bn*(x-x0)*(x-x1)--------(x-xn-1)$$

Data points can be used to evaluate the coefficients b 0 , b 1 , ---- , b n . For an nth-order polynomial, n + 1 data points are required: $[x 0 , f (x 0 )], [x 1 , f (x 1 )],--- , [x n , f (x n )]$. We use these data points and the following equations to evaluate the coefficients:\\
$b 0 = f(x 0 )$\\
$b 1 = f [x 1 , x 0 ]$\\
$b 2 = f [x 2 , x 1 , x 0 ]$\\
-\\
-\\
-\\
$b n = f [x n , x n-1 , . . . , x 1 , x 0 ]$\\

where the bracketed function evaluations are finite divided differences. For example, the first finite divided difference is represented generally as $$f [x i , x j ] =
f (x i ) - f (x j ) / (x i - x j)$$
Similarly, the nth finite divided difference is$$f [x n , x n-1 , ---, x 1 , x 0 ] =
f [x n , x -1 , --- , x 1 ] - f [x n-1 , x n-2 ,-- , x 0 ]/(x n - x 0)$$

These coefficients can be evaluated and put in the equation to yield the interpolation polynomial.
\subsection{LAGRANGE INTERPOLATION METHOD}
The Lagrange interpolating polynomial is simply a reformulation of the Newton polynomial that avoids the computation of divided differences. It can be represented concisely as$$f(x)= \sum_{i=o}^{n} Li(x) *f (x i )$$
where$$Li(x)=\prod_{j=o,i}^{b} f(i)$$
For example, the linear version (n = 1) is$$f 1 (x) =(x - x 1)*f(x 0 )/(x 0 - x 1)+(x - x 0)*f(x 1 )/(x 1 - x 0)$$
the second-order version is$$f 2 (x) =(x - x 1 )*(x - x 2 )*f(x 0 )/(x 0 - x 1 )(x 0 - x 2 )+(x - x 0 )*(x - x 2 )*f(x 1 )/(x 1 - x 0 )(x 1 - x 2 )$$ $ +(x - x 0 )*(x - x 1 ) *f(x 2 )/(x 2 - x 0 )(x 2 - x 1 )$

\section{ALGORITHM:}
\subsection{NEWTON'S DIVIDED DIFFERENCE}
\section{PROGRAM CODE:}
\subsection{LAGRANGE INTERPOLATION METHOD}
Main Program:\\
FUNCTION: $f(x)=1/(1+25x^2)$
\begin{lstlisting}
program Mainfunction1 !program start
use Newton
use lagrange  !to call the module in the main program
implicit none!to force the programmer to define all the variables

real(8),dimension(0:4)::X4,Y14 !(X4:Datapoints for 4 order,Y14: Function values at datapoints for 1 problem,Y24: Function values at datapoints for 2 problem )
real(8),dimension(0:9)::X9,Y19,Lagint9,Lagint4,XX,Lagrangey1,error14,error19&
,Newint4,error14n,Newint9,error19n !(X9:Datapoints for 9 points,Y19: Function values at datapoints for 1 problem,Y29: Function values at datapoints for 2 problem, Lagint9:inerpolated value for 9 order,Lagint4:interpolated value for 4 order,Lagrange:points at which interpolation is done,Lagrangey:function values at data points to calulate error,error14:error in 1 problem and 4 order,error19:error in 1 problem and 9 order,error24:error in 2 problem and 4 order,error29:error in 2 problem and 9 order )
integer::i,j,N,k
character(len=30)::myfilename
real(8)::A,B,H
real(8), parameter::pi=3.14

B=1 !Max range
A=-1!min range

X4(:)=(/-1.0,-0.5,0.0,0.5,1.0/)                                              !datapoints for 4th order
X9(:)=(/-1.0,-0.8,-0.6,-0.4,-0.2,0.0,0.2,0.4,0.6,0.8/)                       !datapoints for 9th order
Y14(:)=1/(1+(25*(X4(:)**2)))                                                 !values defined at data points
Y19(:)=1/(1+(25*(X9(:)**2)))                                                 !values defined at data points
N=9                                                                          !number of points at which the interpolation needs to be done


XX(:)=(/-0.93,-0.77,-0.64,-0.35,-0.23,0.0,0.13,0.26,0.53,0.76/)

do i=0,N
Lagint4(i)=lgrnge(X4,Y14,4,xx(i))
end do
do i=0,N
Newint4(i)=order4(X4,Y14,4,xx(i))
end do
do j=0,N
Lagint9(i)=lgrnge(X9,Y19,9,xx(i))
end do
do j=0,N
Newint9(i)=order4(X9,Y19,9,xx(i))
end do

Lagrangey1=1/(1+(25*(xx(:)**2)))

myfilename='function1.dat'
open(99,file=myfilename)
write(99,*)'for function f(x)=1/(1+25x^2), X VALUES'
write(99,*)xx
write(99,*)'for function f(x)=1/(1+25x^2), Y VALUES'
write(99,*)lagrangey1
write(99,*)'for LAGRANGE'
write(99,*)'4th order'
write(99,*)Lagint4
write(99,*)'9th order'
write(99,*)Lagint9
write(99,*)'for NEWTON'
write(99,*)'for function f(x)=1/(1+25x^2)'
write(99,*)'4th order'
write(99,*)Newint4
write(99,*)'9th order'
write(99,*)Newint9
close(99)


end Program Mainfunction1

	\end{lstlisting}
FUNCTION: $F(X)=cos(x(:)*pi)$ //	
\begin{lstlisting}	
program MainNI !program start
use Newton
use lagrange  !to call the module in the main program
implicit none!to force the programmer to define all the variables

real(8),dimension(0:4)::X4,Y24 !(X4:Datapoints for 4 order,Y14: Function values at datapoints for 1 problem,Y24: Function values at datapoints for 2 problem )
real(8),dimension(0:9)::X9,Lagint9,Lagint4,xx,Lagrangey2,error24,&
error29,Y29,Newint4,error14n,error24n,Newint9,error19n,error29n !(X9:Datapoints for 9 points,Y19: Function values at datapoints for 1 problem,Y29: Function values at datapoints for 2 problem, Lagint9:inerpolated value for 9 order,Lagint4:interpolated value for 4 order,Lagrange:points at which interpolation is done,Lagrangey:function values at data points to calulate error,error14:error in 1 problem and 4 order,error19:error in 1 problem and 9 order,error24:error in 2 problem and 4 order,error29:error in 2 problem and 9 order )
integer::i,j,N,k
character(len=30)::myfilename
real(8)::A,B,H
real(8), parameter::pi=3.14

B=1 !Max range
A=-1!min range

xx(:)=(/-0.93,-0.77,-0.64,-0.35,-0.23,0.0,0.13,0.26,0.53,0.76/)


X4(:)=(/-1.0,-0.5,0.0,0.5,1.0/)                                !datapoints for 4th order
X9(:)=(/-1.0,-0.8,-0.6,-0.4,-0.2,0.0,0.2,0.4,0.6,0.8/)         !datapoints for 9th order
Y24(:)=cos(X4(:)*pi)                                           !values defined at data points
Y29(:)=cos(X9(:)*pi)                                           !values defined at data points
N=9                                                            !number of points at which the interpolation needs to be done

xx(:)=(/-0.93,-0.77,-0.64,-0.35,-0.23,0.0,0.13,0.26,0.53,0.76/)
Lagrangey2(:)=cos(xx(:)*pi) 

do i=0,N
Lagint4(i)=lgrnge(X4,Y24,4,xx(i))
end do

do i=0,N
Newint4(i)=order4(X4,Y24,4,xx(i))
end do

do j=0,N
Lagint9(i)=lgrnge(X9,Y29,9,xx(i))
end do
do j=0,N
Newint9(i)=order4(X9,Y29,9,xx(i))
end do

myfilename='function2.dat'
open(99,file=myfilename)

write(99,*)'for LAGRANGE'
write(99,*)'for function f(x)=cos(x*pi), X VALUES'
write(99,*)xx
write(99,*)'for function f(x)=cos(x*pi), Y VALUES'
write(99,*)lagrangey2
write(99,*)'4th order'
write(99,*)Lagint4
write(99,*)'9th order'
write(99,*)Lagint9
write(99,*)'for NEWTON'
write(99,*)'for function f(x)=cos(x*pi)'
write(99,*)'4th order'
write(99,*)Newint4
write(99,*)'9th order'
write(99,*)Newint9
close(99)


end Program MainNI

	\end{lstlisting}
MODULE LAGRANGE
   \begin{lstlisting}
module lagrange
implicit none
	
contains

function lgrnge(X,Y,N,XX)
real(8),dimension(0:N)::X,Y,Multiply !X: datapoints at which Interpolation needs to be perform, Y: function values at the datapoints
integer::i,j,N !i,j are do loop integers and N is the order of the input.
real(8)::lgrnge,total,XX !interpolated value of the xx input to the function
	
total=0
	
do i=0,N
Multiply(i)=Y(i)
	
do j=0,N
if (j==i) cycle 
Multiply(i)=Multiply(i)*(XX-X(j))/(X(i)-X(j))
end do
total=total+Multiply(i)
end do
lgrnge=total
end function lgrnge
	
end module lagrange
	\end{lstlisting}
		
		
	MODULE NEWTON
	\begin{lstlisting}
module Newton
implicit none
	
contains
!N=number of data point=+1 the order of the polynomial used for interpolation
	
	
function order4(X,Y,N,XX)
real(8),allocatable:: totala(:),roots(:)
real(8),dimension(0:N)::X,Y
real(8),dimension(0:N+1,0:N)::R
real(8)::total,order4,XX
real(8)::multiplier
integer::N,i,j,k,l,m,o,p

allocate(totala(N+1))
	
allocate(roots(N+1))
	
	
do i=2,N+1
do k=0,N
R(0,k)=X(k)
end do 
do l=0,N
R(1,l)=Y(l)
end do
	
do j=0,N
if (j>=N+2-i) then
R(i,j)=0
else
R(i,j)=(R(i-1,j+1)-R(i-1,j))/(R(1,i-3+j+1)-R(1,j))
end if
end do
end do
	
do m=1,N+1
roots(m)=R(1,m)
end do
	
	
do o=2,N+1
multiplier=roots(o)
totala(1)=roots(1)
do p=2,o+1
totala(o)=multiplier*(XX-X((p-1)))
end do
end do 
order4=sum(totala)
	
end function

end module Newton
\end{lstlisting}


\section{RESULT:}
For	f(x)=1/(1+25x2)\\
        \begin{center}
        	\begin{tabular}{|c| c| c| c| c| c| c|}
        			
        	 
        		\hline
					
		S.No.	&XX	&YY	&NEWINT4	&NEWINT9 	&LAGINT4	&LAGINT9\\
		\hline
		1	&-0.9300000072	&0.0442037788	&-2.6470587841676605	&4.42E-02	&3.8461538461538464E-002	&0.0000000000000000 \\
		\hline
		2	&-0.7699999809	&0.0632011406	&-0.87531827994936018	&6.32E-02	&-0.37039319276961191	&2.6312747813848758E-312\\
		\hline
		3	&-0.6399999857	&0.0889679752	&-0.72238725107607848	&8.90E-02	&-0.19566257088658587	&0.0000000000000000\\
		\hline
		4	&-0.349999994	&0.2461538525	&-0.38123341473881700	&0.2461538525	&0.52579991498700174	&2.6312747813848758E-312\\
		\hline
		5	&-0.2300000042	&0.430570497	&-0.24006631790563979	&0.430570497	&0.78301527761008283	&0.0000000000000000\\
		\hline
		6	&0	&1	&0.9946949602	&3.0503978779840846E-002	&1.0000000000000000 	&4.9406564584124654E-324\\
		\hline
		7	&0.1299999952	&0.702987713	&0.7101839655	&0.702987713	&0.92866250516381754	&6.1074888238675082E-317\\
		\hline
		8	&0.2599999905	&0.371747229	&0.4050835668	&0.371747229	&0.72601381209143256	&6.1076157987384894E-317\\
		\hline
		9	&0.5299999714	&0.1246494353	&8.13E-02	&0.1246494353	&6.0158594442995976E-002	&6.1074260775304864E-317\\
		\hline
		10	&0.7599999905	&0.0647668409	&-0.587082452	&6.48E-02	&-0.36433102800384382	&3.9525251667299724E-323\\
		\hline
	\end{tabular}
\end{center}

		
	For 	f(x)=cos(pi*x)	\\
		 \begin{center}
		 	\begin{tabular}{|c| c| c| c| c| c| c|}
		 		
				\hline
		S.No.	&XX	&YY	&NEWINT4	&NEWINT9 	&LAGINT4	&LAGINT9\\
		\hline
		1	&-0.9300000072	&-0.9755925897	&-9.76E-01	&-9.76E-01	&-0.987357316	&-0.9755926111\\
		\hline
		2	&-0.7699999809	&-0.7492994708	&-1.45E+00	&-7.49E-01	&0.1260187557	&-0.7492995243\\
		\hline
		3	&-0.6399999857	&-0.4248567412	&-0.6604939021	&-4.25E-01	&0.2975304491	&-0.424856802\\
		\hline
		4	&-0.349999994	&0.4544871185	&0.4969066814	&0.4544870858	&0.8357205005	&0.4544870858\\
		\hline
		5	&-0.2300000042	&0.750353256	&0.7778651441	&0.7503532401	&0.990419558	&0.7503532401\\
		\hline
		6	&0	&1	&0.9946949602	&1	&0.9694960212	&1\\
		\hline
		7	&0.1299999952	&0.9178368394	&0.9250330865	&0.917836834	&0.7344018264	&0.917836834\\
		\hline
		8	&0.2599999905	&0.6848489276	&0.7181852455	&0.6848489077	&0.3484828712	&0.6848489077\\
		\hline
		9	&0.5299999714	&-0.0932678302	&-1.37E-01	&-9.33E-02	&-0.7472598944	&-9.33E-02\\
		\hline
		10	&0.7599999905	&-0.7281394858	&-1.3799888334	&-7.28E-01	&-1.6527018635	&-0.7281395405\\
		\hline
	\end{tabular}
\end{center}








\end{document}